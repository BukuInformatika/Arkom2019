\section{Arsitektur dan fungsi CPU}
\subsection{Pengertian CPU} 
CPU/Central Processing Unit adalah perangkat keras komputer yang mempunyai fungsi untuk menerima dan melakukan perintah dan data dari perangkat lunak. Karena merupakan pusat pengolahan data dalam sebuah komputer, CPU sering disebut sebagai processor. Cepat atau lambatnya kinerja dari sebuah komputer salah satunya dapat dilihat dari kualitas dan teknologi dari CPU yang digunakan.

\subsection{KOMPONEN UTAMA CPU}
Arihtmetic Logikal Unit (ALU). Fungsinya :
\begin{enumerate}
\item Melakukan komputasi untuk pengolahan data.
\item Melakukan tugas-tugas dasar aritmatik dan operasi logika.
\end{enumerate}

Control Unit. Fungsinya : 
\begin{enumerate}
\item  Mengatur dan mengendalikan alat-alat masukan (input) dan keluaran (output).
\item Mengambil instruksi-instruksi dari memori utama.
\item Mengambil data dari memori utama (jika diperlukan) untuk diproses.
\item Mengirim instruksi ke ALU apabila ada perhitungan aritmatika atau perbandingan logika serta mengawasi kerja dari ALU.
\item Menyimpan hasil proses ke memori utama.
\end{enumerate}

Register, fungsinya :
Memori internal yang didesain untuk dapat menyimpan data lebih cepat dibandingkan memori utama

Internal Bus, fungsinya :
Jalur yang berfungsi sebagai jembatan komunikasi antara komponen utama


