\section{Definisi}
\par Arsitektur komputer adalah suatu konsep perencanaan dan juga struktur pengoperasian dasar dari suatu sistem komputer atau ilmu yang bertujuan untuk perancangan sistem komputer. Arsitektur komputer dapat dikategorikan sebagai ilmu sekaligus sebuah seni mengenai cara interkoneksi antara berbagai komponen perangkat keras atau hardware untuk dapat menciptakan sebuah komputer yang dapat memenuhi kebutuhan fungsional, kinerja, dan juga target biaya dalam bidang teknik komputer.
\par Arsitektur  von Neumann (atau Mesin Von Neumann) adalah arsitektur yang diciptakan oleh John von Neumann [1903 – 1957]. Arsitektur ini digunakan oleh hampir pada semua komputer pada saat ini. Arsitektur Von Neumann ini menggambarkan komputer dengan 4 (empat) bagian utama, yaitu: Unit Aritmatika dan Logis (ALU), unit kontrol, memori, dan alat masukan dan hasil (secara kolektif dinamakan I atau O). Bagian tersebut dihubungkan oleh berkas kawat, “bus”.
\par Arsitektur komputer merupakan suatu hal yang sangatlah penting karena dapat memberikan berbagai atribut-atribut pada sistem komputer, hal tersebuti tentunya sangat dibutuhkan bagi perancang ataupun user software sistem dalam mengembangkan suatu program.

\par Arsitektur komputer memiliki 2 bagian utama yaitu:
\begin{itemize}
\item Instructure Set Architecture
\par Instructure Set Architecture (ISA) adalah spesifikasi yang menentukan bagaimana programmer bahasa mesin berinteraksi dengan komputer.
\item Hardware System Architecture
\par Hardware Set Architecture (HSA) adalah subsistem hardware (perangkat keras) dasar yaitu CPU, Memori, serta OS.

\end{itemize}


\section{Sejarah}
Sejarah perkembangan arsitektur komputer telah dimulai sejak masa perang dunia kedua pada tahun (1945-1955) sebagai generasi pertama.
\begin{enumerate}
\item Generaasi Pertama (1945 - !955)
\par Negara-negara maju yang sedang berperang berlomba-lomba menciptakan peralatan canggih yang digunakan untuk media informasi dan radar  untuk keperluan militer. Komputer diperkenalkan pertama kali di universitas Pensylvania dengan berbasis teknologi tabung hampa udara  yang digunakan pada peralatan radio. Konsep utama arsitektur komputer diperkenalkan oleh john Von Neuman,
\par Program dan datanya diletakkan dalam memori yang sama , operasi aritmatika dasar dilakukan dalam beberapa milidetik menggunakan teknologi tabung hampa udara untuk menerapkan fungsi logika, teknologi ini menghasilkan peningkatan kecepatan  dengan kelipatan 100 hingga 1000 kali relatif terhadap teknologi mekanik dan elektro mekanik berbasis relay dan  fungsi I/O dilaksanakan oleh alat yang mirip mesin ketik .
\end{enumerate}




\section{Software dan Hardware}
Perintah navigasi direktori

