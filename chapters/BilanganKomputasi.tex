\section{Biner}
\subsection{Pengertian Bilangan Biner atau Binary}
Bilangan biner atau bisa juga disebut bilangan binary merupakan sistem penulisan angka dengan hanya menggunkan dua simbol
yaitu 1 dan 2. bilangan biner merupakan dasardari semua sistem bilangan yang berbasis digital. dari sistem biner kita dapat
mengkonversikannya ke sistem bilangan Oktal atau Hexadesimal.

Bilangan biner umumnya digunakan dalam dunia komputasi. komputer menggunakan bilangan biner agar dapat saling berinteraksi
terhadap semua komponen (hardware) dan bisa juga berinteraksi terhadap sesama komputer. contoh nya pada sebuah komputer yaitu
apabila sebuah komputer terhubung dengan tegangan listrik maka bernilai 1 dan apabila komputer tidak terhubung dengan jaringan
listrik makanilai nya 0.

operasi bilangan biner  adalah operasi antara dua bilangan. dasar perkalian adalah tabel yang memuat hasil perkalian operasi
pada biner antara bilangan satu digit.

\subsection{Bilangan Biner}
\qquad Sebagai perumpamaan untuk bilangan desimal, untuk angka 157 : $157_{(10)}$ = (1 x 100) + (5 x 10) + (7 x 1) \\

Perhatikan! Bilangan desimal atau sering juga disebut dengan basis 10. Hal ini dikarenakan perpangkatan 10 yang didapat dari 100, 101, 102, dst.

\subsection{Mengenal Konsep Bilangan Biner dan Desimal}
\qquad Perbedaan paling mendasar dari metode bilangan biner dan bilangan desimal terletak pada jumlah dari basisnya. Jika desimal berbasis 10 (x10) berpangkatkan 10x, maka untuk bilangan biner berbasiskan 2 (x2) menggunakan perpangkatan 2x.
Sederhananya perhatikan contoh dibawah ini!\\
Untuk Desimal:
\begin{table}[!ht]
\begin{tabular}{ l l }
$14_{(10)}$ & = (1 x $10^1$) + (4 x $10^0$)\\
& = 10 + 4\\
& = 14\\
\end{tabular}
\end{table}
\\
Untuk Biner:
\begin{table}[!ht]
\begin{tabular}{ l l }
$1110_{(2)}$ & = (1 x $2^3$) + (1 x $2^2$) + (1 x $2^1$) + (0 x $2^0$)\\
& = 8 + 4 + 2 + 0\\
& = 14\\
\end{tabular}
\end{table}

Bentuk umum dari bilangan biner dan bilangan desimal bisa dilihat pada tabel \ref{table:binerdesimal}.

\begin{table}[!ht]
\centering
\begin{tabular}{ |c|c|c|c|c|c|c|c|c|c| }
\hline
Biner & 1 & 1 & 1 & 1 & 1 & 1 & 1 & 1 & 11111111 \\
\hline
Desimal & 128 & 64 & 32 & 16 & 8 & 4 & 2 & 1 & 255 \\
\hline
Pangkat & $2^7$ & $2^6$ & $2^5$ & $2^4$ & $2^3$ & $2^2$ & $2^1$ & $2^0$ & $X^{1-7}$ \\
\hline
\end{tabular}
\caption{Tabel bentuk umum dari bilangan biner dan bilangan desimal}
\label{table:binerdesimal}
\end{table}


\section{Hexadecimal}
Hexadecimal adalah sebuah sistem bilangan yang menggunakan sebuah simbol.Dalam hexadecimal Terdapat beberapa simbol yang bisa digunakan di sistem bilangan ini.Berbeda dengan bilangan decimal.hexadecimal menggunakan angka 0 sampai 1, di bilangan hexadecimal ini tidak menggunakan angka semua melainkan ada beberapa simbol yang menggunakan huruf.jumlah simbol yang yang berasal dari angka 1 sampai 9 berjumlah 16 simbol, ditambah dengan 6 simbol lainnya yang menggunakan huruf dari A sampai F.Hexadecimal bisa digunakan untuk menampilkan nilai alamat memori dan pemrograman komputer.Teknik penjumlahan dan pengurangan pada bilangan hexadecimal hampir sama dengan penjumlahan dan pengurangan pada bilangan biner,octal dan decimal, tetapi jika terjadi carry 1 atau borrow 1, maka angka 1 tersebut bernilai 16. Carry akan terjadi apabila penjumlahan lebih dari 15 misalnya 8+8=10. Sedangkan borrow terjadi apabila angka yang dikurangi lebih kecil dari pengurang, misalnya 45-6=. 