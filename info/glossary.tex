
\term{Hardware}Merupakan segala piranti atau komponen dari sebuah komputer yang sifatnya bisa dilihat secara kasat mata dan bisa diraba secara langsung. 

\term{Software}Sekumpulan data elektronik yang disimpan dan diatur oleh komputer, data elektronik yang disimpan oleh komputer itu dapat berupa program atau instruksi yang akan menjalankan suatu perintah.

\term{Compiler}Merupakan sebuah program komputer yang berguna untuk menerjemahkan program komputer yang ditulis dalam bahasa pemrograman tertentu menjadi program yang ditulis dalam bahasa pemrograman lain.

\term{Transistor}Merupakan alat semikonduktor yang dipakai sebagai penguat, sebagai sirkuit pemutus dan penyambung (switching), stabilisasi tegangan, modulasi sinyal atau sebagai fungsi lainnya.

\term{Microprogramming}Microprogramming adalah cara pengoperasian bagian kontrol komputer yang menguraikan setiap instruksi menjadi beberapa tahap kecil (microstep) yang merupakan bagian mikroprogram. Sejumlah sistem menyediakan mikroprogram, sehingga pemakai dapat menyesuaikan perintah dengan mesinnya.

\term{Magnetic}Sebuah objek yang mempunyai medan magnet.

\term{Relay}Merupakan suatu peranti yang bekerja berdasarkan elektromagnetik untuk menggerakan sejumlah kontaktor yang tersusun atau sebuah saklar elektronis yang dapat dikendalikan dari rangkaian elektronik lainnya dengan memanfaatkan tenaga listrik sebagai sumber energinya.

\term{Mainframe}Merupakan istilah Teknologi Informasi dalam bahasa Inggris yang mengacu kepada kelas tertinggi dari komputer yang terdiri dari komputer-komputer yang mampu melakukan banyak tugas komputasi yang rumit dalam waktu yang singkat.

\term{Processor}Adalah komponen komputer yang merupakan sebagai otak yang menjalankan proses dan pengendali kerja komputer dengan bekerjasama perangkat komputer lainnya, satuan kecepatan dalam Prosesor adalah Mhz (Mega Heartz) atau Ghz (Giga Heartz) dengan semakin besar kecepatan suatu Prosesor maka akan semakin cepat kinerja komputer saat melakukan proses.